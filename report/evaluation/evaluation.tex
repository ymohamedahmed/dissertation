\section{Face tracking}
\label{section:face_tracking}
\paragraph{Research question}
\paragraph{Metrics}
\paragraph{Results}
% \paragraph{Performance cost}
% The performance of the above algorithm, clearly depends on the proportion that each of the two branches are executed.
% \paragraph{Motion resistance}

% \paragraph{Correctness}

\section{Region selection}
\label{section:region_selection}
\paragraph{Research question}
\paragraph{Metrics}
\paragraph{Results}

\subsection{Skin tone detection}
\label{section:skin_tone_detection}
%Evaluate using the basic skin tone range vs k-means vs heuristic
\paragraph{Research question}
\paragraph{Metrics}
\paragraph{Results}

\section{Heart rate isolation}
\subsection{Pulse isolation}
\label{section:bss}
% ICA vs PCA
%The PCA approach comes with the benefit of identifying the pulse signal as a natural part of the algorithm.
\label{section:ica_assumption}
% evaluate correctness of assumption of maximum power from ICA applied to each signal
\subsection{Heart rate identification}
% \subsection{Blind source separation}
\subsection{Window and stride size}
\paragraph{Research question}
\paragraph{Metrics}
\paragraph{Results}

\section{Evaluation method}
\subsection{Experimental setup}
\subsection{Ethics}

\section{Capabilities of remote heart rate sensing}
% \section{Performance}
\subsection{Accuracy}
% cover stationary and exercise videos
\paragraph{Research question}
\paragraph{Metrics}
\paragraph{Results}

\subsection{Energy consumption}
\paragraph{Research question}
\paragraph{Metrics}
\paragraph{Results}

\subsection{Time cost}
\paragraph{Research question}
\paragraph{Metrics}
\paragraph{Results}

\section{Summary}