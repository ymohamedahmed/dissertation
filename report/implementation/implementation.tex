\section{Overview}
% In this chapter I will outline the implementation 

GIVE STRUCTURE OF PROGRAM AND EXPLAIN WHY MOST WORK IS DEDICATED TO REGION
SELECTION AND HEART RATE ISOLATION SPECIFICALLY

THAT IS FACE DETECTION IS MOSTLY A SOLVED PROBLEM
BACKGROUND ON HOW DESIGN IS TO MIMIC A PPG SENSOR

\section{System design}
The system was designed with a focus on decomposition. Abstractly, the program consists of three distinct tasks, each of which rely on the result from the previous. Together, forming a kind of pipeline. 
\begin{itemize}
    \item \textbf{Face detection}: identify a face in each supplied camera frame
    \item \textbf{Region selection}: given a bounding box around a face, select some set of pixels to consider 
    \item \textbf{Heart rate isolation}: given a time series of the mean colour of some region of the face, infer the heart rate
\end{itemize}
Each of these occur at different rates and, thereby, have different performance constraints which must be held.
For example, face detection and region selection operate per frame and so should run in real-time as to minimise the number of dropped frames.
Heart rate isolation, however, is only executed after some adequate number of data points is received, denoted as the \textbf{window size} and is recomputed every so often. The specifics of thi
Separating these tasks and connecting them through minimal interfaces helps to minimise interdependence which subsequently informs

\subsection{Asynchronous execution}
TALK ABOUT CONCURRENCY RUNNING HR ISOLATION ASYNCHRONOUSLY
\section{Face detection}
TALK ABOUT CRITICAL NOT TO DROP FRAMES
\subsection{Repeated face detection}
\subsection{Point tracking}
\subsection{Face alignment}
%As opposed to 

\section{Region of interest selection}
DEFINE THE BASELINE: PRIMITIVEROI
DISCUSS THE FUNDAMENTAL TRADEOFF BETWEEN NUMBER OF PIXELS AND FIDELITY
\subsection{Facial landmarks}
\subsection{Skin detection}
Face detection systems, typically, return a bounding box, within which it is believed
a face is present. However, naturally, the box will also contain pixels from the background of 
the image since faces are not, in general, perfectly rectangular.
These background pixels will not contain any information as to the underlying heart rate of the user.
As a result, considering the entire bounding box will add unecessary noise to the resulting signal.
The ideal algorithm would only consider skin pixels, however, robust, pose-invariant skin detection is an unsolved problem.
On the other hand, considering too few pixels could increase the effect of specular reflection.

\subsubsection{Clustering approaches}
\paragraph{K-Means}
TALK ABOUT COMPLEXITY OF K-MEANS AND WHY USING IT REPEATEDLY DOESN'T WORK THAT WELL
%\subsubsection{Hierarchical clustering}
%\subsubsection{Markov clustering}

%\subsection{}
%\subsection{Flood filling}
\subsection{Colour Space Filtering}
EXPLAIN PERCEPTUAL UNIFORMITY 
SHOW YCbCr REPRESENTATION OF SKIN PIXELS
%\subsection{Conditional Random Fields}

% \section{Ensemble Methods}

\section{Heart rate isolation}
EXPLAIN WHY IT'S NOT JUST THE FOURIER POWER SPECTRUM
IN VIDEOS WITH MOVEMENT WE EXPECT HEART RATE TO BE A SERIES OF PEAKS TOGETHER RATHER THAN A SINGLE PEAK
THAT MIGHT BE DUE TO LIGHTING ETC
IMPACT OF LIGHTING CONDITIONS

\subsection{Blind-source separation}
ASSUMPTION OF CAMERA FEED BEING A MIXTURE OF INDEPENDENT SOURCES INCLUDING HEART RATE

SELECTION OF RESULTING COMPONENT => HIGHEST PEAK

%SHOW IDEALISED HR AND WHY AUTOCORRELATION HELPS US PICK IT OUT


\subsection{Respiration rejection}

\subsection{Optimisation}
%\subsection{}

\section{Repository overview}